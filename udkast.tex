\documentclass[10pt]{article}

\usepackage{titlesec}
\usepackage{multicol}
%\usepackage{enumerate}
\usepackage[shortlabels]{enumitem}
\usepackage{amsmath}
\usepackage{soul}

\title{Vedtægter for Fagrådet Sigma\\
\large{23. maj 2023}}

\author{}

\titleformat{\section}{\normalfont\Large\bfseries}{\S\thesection}{1em}{}

\titleformat{\subsection}[runin]{\normalfont\normalsize\sffamily}{\textsl{Stk. \thesubsection}.}{0.5em}{}

\renewcommand{\thesubsection}{\arabic{subsection}}

%\newcommand{\stk}

\newcommand{\fagr}{\text{Fagrådet Sigma}}
\newcommand{\fagrs}{\text{Fagrådet Sigmas}}

\begin{document}
\maketitle
%\begin{multicols}{2}

\sffamily

\section{Navn og hjemsted}\label{navn}

\subsection{Foreningens navn er \fagr{}.}\label{navn:navn}

\subsection{\fagr{} har hjemsted på Ny Munkegade 116 1, 8000 Aarhus C, lokale 1535-129, herefter omtalt Rådet, ved Aarhus Universitet.}\label{navn:hjemsted}

\subsection{Foreningen tegnes af forpersonen og kassereren.}

\section{Formål og tilhørsforhold}\label{formaal}

\subsection{\fagr{} repræsenterer studerende indskrevet på følgende institutter og centre, herefter betegnet som Faggruppen:}

\begin{enumerate}[a), nosep]
\item Institut for Matematik,
\item Institut for Fysik og Astronomi,
\item Institut for Datalogi,
\item Institut for Kemi og
\item Interdisciplinary Nanoscience Center.
\end{enumerate}

\subsection{\fagr{} har til formål at virke til gavn for de studerende i Faggruppen ved at varetage deres fælles faglige og studierelaterede forhold og interesser.}\label{formaal:repr}

\subsection{\fagr{} uddgør Faggruppens repræsentation ved Studenterrådet ved Aarhus Universitet og lader sig derfor repræsentere i Studenterrådets fællesråd.}

\subsection{\fagr{} opstiller kandidater til relevante udvalg og nævn, herunder Studienævnet og Akademisk Råd ved Faculty of Natural Sciences.}\label{formaal:opstilling}

\subsection{Kandidater valgt i henhold til stk. \ref{formaal:opstilling} forpligter sig til at holde \fagr{} orienteret om deres virke og desuden bestræbe sig efter at repræsentere de studerendes interesser jf. stk. 2 i samarbejde med \fagr{} og dets holdninger som vedtaget på møder, generalforsamlinger og studentermøder.}

\section{Generalforsamling}\label{gf}

\subsection{Generalforsamlingen er \fagr{}s øverste myndighed.}

\subsection{Den ordinære generalforsamling afholdes i andet kvartal og indkaldes af bestyrelsen med senest 21 dages varsel på \fagr{}s opslagstavle foran Matematisk Kantine og på \fagr{}s Facebook-side. Indkaldelse til den ordinære generalforsamling skal indeholde digital henvisning til det reviderede regnskab.}\label{gf:ordindkaldelse}

\subsection{Ekstraordinær generalforsamling kan til enhver tid indkaldes med 14 dages varsel, når to tredjedeles flertal af de indvalgte stiller krav herom. Der indgives skriftlig begæring til forpersonen, som herefter forestår indkaldelse i overensstemmelse med stk. \ref{gf:rettidigt}.}\label{gf:ekstraindkaldelse}

\subsection{Generalforsamlingen er beslutningsdygtig, når den er indkaldt rettidigt, og dagsordenen er offentliggjort senest 7 dage før generalforsamlingens afholdelse på \fagr{}s opslagstavle foran Matematisk Kantine og på \fagr{}s Facebook-side.}\label{gf:rettidigt}

\subsection{Generalforsamlingen træffer beslutninger ved simpelt flertal. Dog kræver vedtægtsændringer to tredjedeles flertal. Enhver studerende under Faggruppen har tale- og stemmeret.}

\subsection{Kun studerende under Faggruppen kan opstille til valg i henhold til §\ref{gfi}, stk. \ref{gfi:dagsorden}, nr. 5, litra a-e. Opstilling kan ske fysisk umiddelbart inden valget eller på forhånd ved skriftlig tilkendegivelse til forpersonen.}\label{gf:studerende}

\subsection{Valg i henhold til §\ref{gfi}, stk. \ref{gfi:dagsorden}, nr. 5 afholdes ved simpelt flertal via håndsoprækning. En eller flere stemmeberettigede kan begære skriftlig og hemmelig afstemning.}\label{gf:menigtindvalgte}

\subsection{En person, der er valgt til én post i henhold til §\ref{gfi}, stk. \ref{gfi:dagsorden}, nr. 5 kan ikke vælges til andre poster. Valgte i henhold til §\ref{gfi}, stk. \ref{gfi:dagsorden}, nr., litra a-e, benævnes indvalgte.}\label{gf:begrpost}

\subsection{Der skelnes mellem poster valgt i henhold til stk. \ref{gfi:dagsorden}, nr. 5, litra a-c, som udgør bestyrelsen jf. §\ref{bestyrelsen}, og stk. \ref{gfi:dagsorden}, nr. 5, d-e, som udgør menige medlemmer.}

\subsection{Dirigenten orienterer Studenterrådets Dirigentinstitution om generalforsamlingens valg af posterne beskrevet i §\ref{gfi}, stk. \ref{gfi:dagsorden}, nr. 5, litra a-c.}\label{gf:orienterer}

\section{Generalforsamlingens indhold}\label{gfi}

\subsection{Dagsorden for den ordinære generalforsamling skal indeholde mindst følgende punkter:}\label{gfi:dagsorden}

\begin{enumerate}[1), nosep]
\item Formalia

	\begin{enumerate}[a., nosep]
	\item Valg af dirigent
	\item Konstatering af beslutningsdygtighed
	\item Valg af referent
	\item Valg af stemmetællere
	\end{enumerate}
\item Forpersonens beretning
\item Kassererens beretning
\item Studenterrådets beretning
\item Valg af
	\begin{enumerate}[a., nosep]
	\item Forperson
	\item Næstforperson
	\item Kasserer
	\item Intern revisor
	\item Menigt indvalgte
	\item Ekstern revisor
	\end{enumerate}
\item Indkomne forslag
\item Dato for første møde i \fagr{}
\item Eventuelt
\end{enumerate}

\subsection{På en ekstraordinær generalforsamling behandles altid punkterne i stk. \ref{gfi:dagsorden}, nr. 1, 6 og 8. Øvrige punkter fra stk. \ref{gfi:dagsorden} samt punkterne i øvrigt kan behandles, såfremt disse nævnes i begæringen. Der kan kun behandles indkomne forslag, som er indeholdt i begæringen.}\label{gfi:ekstra}

\subsection{Forpersonens beretning, jf. stk. \ref{gfi:dagsorden}, nr. 2, skal som minimum indeholde oversigten over afholdte arrangementer i foreningens regi og andre politiske aktiviteter.}

\subsection{Kassererens beretning, jf. stk. \ref{gfi:dagsorden}, nr. 3, skal som minimum indeholde fremlæggelse af det reviderede regnskab til godkende samt revisorernes fremlæggelse af bemærkninger til regnskabet.}

\subsection{Studenterrådets beretning, jf. stk. \ref{gfi:dagsorden}, nr. 4, skal som minimum indeholde en oversigt over de vigtigste arrangementer i Studenterrådet med relevans for Faggruppen.}

\subsection{Indkomne forslag, jf. stk. \ref{gfi:dagsorden}, nr. 6, kan indsendes af enhver studerende under Faggruppen og skal være forpersonen i hænde senest 14 dage før generalforsamlingens afholdelse.}

\section{Studentermøder}

\subsection{Studentermødet er Fagrådet Sigmas næstøverste myndighed.}

\subsection{På studentermøder har alle studerende ved Faggruppen tale- og stemmret. De indvalgte kan herudover beslutte at give andre deltagere taleret.}

\subsection{Studentermødet afholdes efter behov og indkaldes af bestyrelsen med mindst 21 dages varsel på Fagrådet Sigmas opslagstavle foran Matematisk Kantine og på Fagrådet Sigmas Facebook-side. Studentermødet er beslutningsdygtigt, når det er rettidigt indkaldt.}

\subsection{Dagsordener på studentermødet skal som minimum indeholde:}

\begin{enumerate}[1), nosep]
\item Formalia
	\begin{enumerate}[a., nosep]
	\item Valg af dirigent blandt de indvalgte
	\item Konstatering af beslutningsdygtighed
	\item Valg af referent blandt de indvalgte
	\end{enumerate}
\item Forpersonens beretning om \fagr{}s arbejde
\item Debat
\item Eventuelt
\end{enumerate}

\subsection{Beslutninger træffes ved simpelt flertal. Der kan ikke træffes beslutninger under punktet Eventuelt eller punkter, som ikke fremgik på dagsordenen ved indkaldelsen.}\label{dagligledelse:flertal}

\section{Bestyrelsen}\label{bestyrelsen}

\subsection{\fagr{}s bestyrelse udgøres af forpersonen, næstforpersonen og kassereren indvalgt i henhold til §\ref{gfi}, stk. \ref{gfi:dagsorden}, nr. 5, litra a-c.}

%\subsection{Såfremt ingen næstforperson indvælges jf. §\ref{gfi}, stk. \ref{gfi:dagsorden}, nr. 5, litra b, udgøres bestyrelsen af forpersonen og kassereren.}

\subsection{Såfremt et medlem af bestyrelsen udtræder inden den kommende ordinære generalforsamling jf. §\ref{gfi}, stk. \ref{gfi:dagsorden}, har den resterende bestyrelse pligt til at indkalde til en ekstraordinær generalforsamling. jf. §\ref{gfi}, stk. \ref{gfi:dagsorden} inden for 14 dage. §\ref{gfi}, stk. \ref{gfi:dagsorden}, pkt. 5, litra a-c skal da som minimum også være indeholdt.}

\subsection{Bestyrelsen har ansvaret for driften af foreningen og skal handle inden for rammerne af beslutningerne truffet på fagrådsmøder jf. § \ref{dagligledelse}. Bestyrelsen varetager desuden forvaltningen af lokalet Rådet på vegne af \fagr{}, herunder kommunikation med eksterne foreninger omkring brug og rengøring af lokalet samt udlån af og regnskab over lokalenøgler.}

\section{Fagrådsmøder}\label{dagligledelse}

%\section{Foreningens daglige ledelse}\label{dagligledelse}

\subsection{Møder i \fagr{}, herefter omtalt fagrådsmøder, lægger rammerne for \fagr{}s virke.}\label{dagligledelse:daglig}

%\subsection{\fagr{}s daglige ledelse udgøres af møderne i \fagr{}.}\label{dagligledelse:daglig}

\subsection{På fagrådsmøder har indvalgte tale- og stemmeret. Ligeledes har studerende under Faggruppen tale- og stemmeret. De indvalgte kan herudover beslutte at give andre deltagere taleret. Beslutninger træffes ved simpelt flertal.}\label{dagligledelse:stemmeret}

\subsection{Bestyrelsen indkalder til møde med dagsorden senest 2 dage før mødets afholdelse, hvorved mødet er beslutningsdygtigt. Der kan dispenseres fra dette ved afstemning.}\label{dagligledelse:indkaldelse}

\subsection{Dagsordenen skal som minimum indeholde:}\label{dagligeledelse:dagsorden}

\begin{enumerate}[1), nosep]
\item Valg af dirigent og referent
\item Konstatering og beslutningsdygtighed
\item Godkendelse af dagsordenen
\item Meddelelser fra:
	\begin{enumerate}[a., nosep]
	\item Eksterne udvalg, hvor Faggruppen er repræsenteret
	\item Fællesrådet
	\item Udvalg under Fagrådet Sigma
	\item Underforeninger
	\end{enumerate}
\item Valg af menigt indvalgte efter proceduren i § \ref{gf}, stk. \ref{gf:studerende}-\ref{gf:begrpost}
\item Planlægning af næste møde
\item Eventuelt
\end{enumerate}

\subsection{Beslutninger træffes ved simpelt flertal. Der kan ikke træffes beslutninger under punktet Eventuelt eller punkter, som ikke fremgik på dagsordenen ved inkaldelsen,}\label{dagligledelse:beslutninger}

\subsection{Referatet offentliggøres senest 7 dage efter mødet. Referanten kan træffe beslutning om begrænset offentliggørelse af specifikke punkter.}

\section{Økonomi og revision}\label{revision}

\subsection{Kassereren fører løbende regnskab og tilsyn med \fagr{}s økonomi og bistår revisorerne med fremskaffelse af kontoudtog og øvrige bilag til regnskabet.}

\subsection{Udlæg, der overstiger 2.000 kr., skal godkendes på et møde i \fagr{}, jf. §\ref{dagligledelse}.}

\subsection{Den indvalgte interne og eksterne revisor, jf. §\ref{gfi}, stk. \ref{gfi:dagsorden}, punkt 5, litra d og f, betegnes samlet som den kritiske revision.}

\subsection{Den kritiske revision skal inden den ordinære generalforsamling gennemgå regnskabet og kontrollere, at det stemmer overens med beholdningerne. Den kritiske revision påtegner regnskabet. Kassereren har pligt til at indsende regnskabet til den kritiske revision senest 7 dage før generalforsamlingen.}\label{revision:kritiske}

\section{Underforeninger}\label{underforeninger}

\subsection{En underforening af \fagr{} er en forening nævnt i stk. \ref{underforeninger:liste}, som repræsenterer en eller flere uddannelser i Faggruppen, og som i sine vedtægter forpligter sig til bestemmelser i nærværende paragraf. Medlemmer af en underforening er ikke nødvendigvis medlemmer i \fagr{}.}

\subsection{Følgende foreninger er underforeninger af \fagr{}:}\label{underforeninger:liste}

\begin{enumerate}[a), nosep]
\item Institut for Fysik \& Astronomi StudenterRepræsentation ("ISR")
\end{enumerate}

\subsection{Underforeninger sender mødereferater til \fagr{} og underretter regelmæssigt forpersonen i \fagr{} om status i underforeningen.}

\subsection{En underforening kan udmeldes af \fagr{}, hvis det vedtages ved en generalforsamling i underforeningen. For foreninger, der ikke har generalforsamlinger, skal der stemmes for ved et studentermøde og et internt møde i underforeningen. Ved udmeldelse fjernes underforeningen automatisk fra listen i stk. \ref{underforeninger:liste}.}

\section{Opløsning af foreningen}

\subsection{\fagr{} kan opløses, hvis dette besluttes på en ekstraordinær generalforsamling og bekræftes indenfor 28 dage på endnu en ekstraordinær generalforsamling. På begge forsamlinger kræves to tredjedeles flertal.}

\subsection{Ved opløsning af \fagr{} overgår aktiver til Studenterrådet ved Aarhus Universitet med henstilling til at aktiverne skal bruges til gavn for \fagr{} Faggruppe.}

\section*{}

\textit{Ændring af statutten er foretaget på generalforsamlingen d. 27 marts 2021}

\noindent
\textit{Ændring af statutten er foretaget på generalforsamlingen d. 7 december 2022}

\noindent
\textit{Ændring af statutten er foretaget på generalforsamlingen d. 23 maj 2023}

%\section*{Ændringsforslag}

\section*{Nye paragrafer}
\section*{Foreningens bestyrelse}

\subsection{\fagr{}s bestyrelse udgøres af forpersonen, næstforpersonen og kassereren indvalgt i henhold til §\ref{gfi}, stk. \ref{gfi:dagsorden}, nr. 5, litra a-c.}

\subsection{Såfremt ingen næstforperson indvælges jf. §\ref{gfi}, stk. \ref{gfi:dagsorden}, nr. 5, litra b, udgøres bestyrelsen af forpersonen og kassereren.}

\subsection{Såfremt et medlem af bestyrelsen udtræder inden den kommende ordinære generalforsamling jf. §\ref{gf}, stk. \ref{gf:ordindkaldelse}, har den resterende bestyrelse pligt til at indkalde til en ekstraordinær generalforsamling. jf. §\ref{gf}, stk. \ref{gf:ekstraindkaldelse}.}

\subsection{\fagr{}s bestyrelse har ansvaret for den daglige drift af foreningen og at sikre kommunikationen mellem indvalgte og diverse udvalg, nævn og underforeninger, herunder Studenterrådet ved Aarhus Universitet, indvalgte kandidater i Studienævnet og Akademisk Råd ved Faculty of Natural Sciences og underforeninger af \fagr{} som beskrevet i §\ref{formaal:opstilling}, stk. \ref{formaal:repr}-\ref{formaal:opstilling} og §\ref{underforeninger} stk. \ref{underforeninger:liste}.}

\subsection{Bestyrelsen kan ikke agere uden samtykke fra \fagr{}s indvalgte som besluttet til et møde jf. §\ref{dagligledelse}, stk. \ref{dagligledelse:daglig}, \ref{dagligledelse:stemmeret}, \ref{dagligeledelse:dagsorden} og \ref{dagligledelse:flertal}. Bestyrelsen har dog frihed til at varetage kommunikation med organisationer udefra og udarbejde ikke-bindende forslag til beslutninger og aftaler såfremt disse behandles og kan vedtages på et ordinært møde.}

\section*{Paragrafændringer}

\section*{Forslag til §\ref{navn}}

\subsection*{Ændring af stk. \ref{navn:hjemsted}:}

\fagr{} har hjemsted på Ny Munkegade 116 1, 8000 Aarhus C, lokale 1535-129, \textbf{i daglig tale kendt som Rådet, ved Aarhus Universitet}.

\section*{Forslag til §\ref{formaal}}

\subsection*{Ændring til stk. \ref{formaal:gavn}:}

\fagr{} har til formål at virke til gavn for de studerende i Faggruppen ved at varetage deres fælles faglige\st{, økonomiske og sociale} \textbf{og studierelaterede forhold og} interesser.

\subsection*{Nyt stykke: Enhver studerende under Faggruppen kan blive indvalgt til \fagr{} såfremt denne indvælges til en generalforsamling jf. §\ref{gf}, stk. \ref{gf:menigtindvalgte} eller et ordinært møde jf. §\ref{dagligledelse}, stk. \ref{dagligeledelse:dagsorden}, nr. 5.}

\section*{Forslag til §\ref{gf}}

\subsection*{Ændring af stk. \ref{gf:ordindkaldelse}:}

Den ordinære generalforsamling afholdes i \st{fjerde} \textbf{andet} kvartal og indkaldes af forpersonen med senest \st{28} \textbf{14} dages varsel på \fagr{}s opslagstavle foran Matematisk Kantine og på \fagr{}s Facebook-side. Indkaldelse til den ordinære generalforsamling skal indeholde digital henvisning til det reviderede regnskab.

\subsection*{Ændring af stk. \ref{gf:begrpost}}
En person, der er valgt til én post i henhold til §\ref{gfi}, stk. \ref{gfi:dagsorden}, nr. 5 kan ikke vælges til andre poster. Valgte i henhold til §\ref{gfi}, stk. \ref{gfi:dagsorden}, nr., litra a-e, benævnes indvalgte.

\subsection*{Nyt stykke: Valg i henhold til §\ref{gfi}, stk. \ref{gfi:dagsorden}, nr. 5, litra b, bortfalder, såfremt ingen begærer opstilling.}

\subsection*{Nyt stykke: Den indvalgte hhv. interne og ekstern revisor jr. §\ref{gfi} stk. \ref{gfi:dagsorden}, nr. 5, litra d og f, udgør den kritiske revision.}

\section*{Forslag til §\ref{gfi}}

\subsection*{Ændring af stk. \ref{gfi:dagsorden}:\newline}

Dagsorden for den ordinære generalforsamling skal indeholde mindst følgende punkter:

\begin{enumerate}[1), nosep]
\item Formalia

	\begin{enumerate}[a., nosep]
	\item Valg af dirigent
	\item Konstatering af beslutningsdygtighed
	\item Valg af referent
	\item Valg af stemmetællere
	\end{enumerate}
\item Forpersonens beretning
\item Kassererens beretning
\item Studenterrådets beretning
\item Valg af
	\begin{enumerate}[a., nosep]
	\item Forperson
	\item Næstforperson
	\item Kasserer
	\item Intern revisor
	\item Menigt indvalgte
	\item \st{Kritisk} \textbf{Ekstern} revisor
	\end{enumerate}
\item Indkomne forslag
\item Dato for første møde i \fagr{}
\item Eventuelt
\end{enumerate}


\subsection*{Ændring af stk. \ref{gfi:ekstra}:}

På en ekstraordinær generalforsamling behandles altid punkterne i stk. \ref{gfi:dagsorden}, nr. 1, \st{6} \textbf{5} og \st{8} \textbf{6}. Øvrige punkter fra stk. \ref{gfi} samt punkterne i øvrigt kan behandles, såfremt disse nævnes i begæringen. Der kan kun behandles indkomne forslag, som er indeholdt i begæringen.

\section*{Forslag til §\ref{revision}}

\subsection*{Ændring af stk. \ref{revision:kritiske}:}

\textbf{Den kritiske revision} \st{Revisorerne} skal inden den ordinære generalforsamling gennemgå regnskabet og kontrollere, at det stemmer overens med beholdningerne. Den kritiske \st{revisor} \textbf{revision} påtegner regnskabet. \textbf{Kassereren har pligt til at indsende regnskabet til den kritiske revision senest 7 dage før generalforsamlingen.}

%\end{multicols}
\end{document}
