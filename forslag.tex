\section*{Ændringsforslag}

\section*{Nye paragrafer}
\section*{Foreningens bestyrelse}

\subsection{\fagr{}s bestyrelse udgøres af forpersonen, næstforpersonen og kassereren indvalgt i henhold til §\ref{gfi}, stk. \ref{gfi:dagsorden}, nr. 5, litra a-c.}

\subsection{Såfremt ingen næstforperson indvælges jf. §\ref{gfi}, stk. \ref{gfi:dagsorden}, nr. 5, litra b, udgøres bestyrelsen af forpersonen og kassereren.}

\subsection{Såfremt et medlem af bestyrelsen udtræder inden den kommende ordinære generalforsamling jf. §\ref{gf}, stk. \ref{gf:ordindkaldelse}, har den resterende bestyrelse pligt til at indkalde til en ekstraordinær generalforsamling. jf. §\ref{gf}, stk. \ref{gf:ekstraindkaldelse}.}

\subsection{\fagr{}s bestyrelse har ansvaret for den daglige drift af foreningen og at sikre kommunikationen mellem indvalgte og diverse udvalg, nævn og underforeninger, herunder Studenterrådet ved Aarhus Universitet, indvalgte kandidater i Studienævnet og Akademisk Råd ved Faculty of Natural Sciences og underforeninger af \fagr{} som beskrevet i §\ref{formaal:opstilling}, stk. \ref{formaal:repr}-\ref{formaal:opstilling} og §\ref{underforeninger} stk. \ref{underforeninger:liste}.}

\subsection{Bestyrelsen kan ikke agere uden samtykke fra \fagr{}s indvalgte som besluttet til et møde jf. §\ref{dagligledelse}, stk. \ref{dagligledelse:daglig}, \ref{dagligledelse:stemmeret}, \ref{dagligeledelse:dagsorden} og \ref{dagligledelse:flertal}. Bestyrelsen har dog frihed til at varetage kommunikation med organisationer udefra og udarbejde ikke-bindende forslag til beslutninger og aftaler såfremt disse behandles og kan vedtages på et ordinært møde.}

\section*{Paragrafændringer}

\section*{Forslag til §\ref{navn}}

\subsection*{Ændring af stk. \ref{navn:hjemsted}:}

\fagr{} har hjemsted på Ny Munkegade 116 1, 8000 Aarhus C, lokale 1535-129, \textbf{i daglig tale kendt som Rådet, ved Aarhus Universitet}.

\section*{Forslag til §\ref{formaal}}

\subsection*{Ændring til stk. \ref{formaal:gavn}:}

\fagr{} har til formål at virke til gavn for de studerende i Faggruppen ved at varetage deres fælles faglige\st{, økonomiske og sociale} \textbf{og studierelaterede forhold og} interesser.

\subsection*{Nyt stykke: Enhver studerende under Faggruppen kan blive indvalgt til \fagr{} såfremt denne indvælges til en generalforsamling jf. §\ref{gf}, stk. \ref{gf:menigtindvalgte} eller et ordinært møde jf. §\ref{dagligledelse}, stk. \ref{dagligeledelse:dagsorden}, nr. 5.}

\section*{Forslag til §\ref{gf}}

\subsection*{Ændring af stk. \ref{gf:ordindkaldelse}:}

Den ordinære generalforsamling afholdes i \st{fjerde} \textbf{andet} kvartal og indkaldes af forpersonen med senest \st{28} \textbf{14} dages varsel på \fagr{}s opslagstavle foran Matematisk Kantine og på \fagr{}s Facebook-side. Indkaldelse til den ordinære generalforsamling skal indeholde digital henvisning til det reviderede regnskab.

\subsection*{Ændring af stk. \ref{gf:begrpost}}
En person, der er valgt til én post i henhold til §\ref{gfi}, stk. \ref{gfi:dagsorden}, nr. 5 kan ikke vælges til andre poster. Valgte i henhold til §\ref{gfi}, stk. \ref{gfi:dagsorden}, nr., litra a-e, benævnes indvalgte.

\subsection*{Nyt stykke: Valg i henhold til §\ref{gfi}, stk. \ref{gfi:dagsorden}, nr. 5, litra b, bortfalder, såfremt ingen begærer opstilling.}

\subsection*{Nyt stykke: Den indvalgte hhv. interne og ekstern revisor jr. §\ref{gfi} stk. \ref{gfi:dagsorden}, nr. 5, litra d og f, udgør den kritiske revision.}

\section*{Forslag til §\ref{gfi}}

\subsection*{Ændring af stk. \ref{gfi:dagsorden}:\newline}

Dagsorden for den ordinære generalforsamling skal indeholde mindst følgende punkter:

\begin{enumerate}[1), nosep]
\item Formalia

	\begin{enumerate}[a., nosep]
	\item Valg af dirigent
	\item Konstatering af beslutningsdygtighed
	\item Valg af referent
	\item Valg af stemmetællere
	\end{enumerate}
\item Forpersonens beretning
\item Kassererens beretning
\item Studenterrådets beretning
\item Valg af
	\begin{enumerate}[a., nosep]
	\item Forperson
	\item Næstforperson
	\item Kasserer
	\item Intern revisor
	\item Menigt indvalgte
	\item \st{Kritisk} \textbf{Ekstern} revisor
	\end{enumerate}
\item Indkomne forslag
\item Dato for første møde i \fagr{}
\item Eventuelt
\end{enumerate}


\subsection*{Ændring af stk. \ref{gfi:ekstra}:}

På en ekstraordinær generalforsamling behandles altid punkterne i stk. \ref{gfi:dagsorden}, nr. 1, \st{6} \textbf{5} og \st{8} \textbf{6}. Øvrige punkter fra stk. \ref{gfi} samt punkterne i øvrigt kan behandles, såfremt disse nævnes i begæringen. Der kan kun behandles indkomne forslag, som er indeholdt i begæringen.

\section*{Forslag til §\ref{revision}}

\subsection*{Ændring af stk. \ref{revision:kritiske}:}

\textbf{Den kritiske revision} \st{Revisorerne} skal inden den ordinære generalforsamling gennemgå regnskabet og kontrollere, at det stemmer overens med beholdningerne. Den kritiske \st{revisor} \textbf{revision} påtegner regnskabet. \textbf{Kassereren har pligt til at indsende regnskabet til den kritiske revision senest 7 dage før generalforsamlingen.}
